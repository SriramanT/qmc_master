\documentclass[10pt]{article}
\usepackage[T1]{fontenc}
\usepackage{graphicx}
\usepackage{bm}
\usepackage{geometry}
\usepackage{bbm}
\geometry{a4paper,total={170mm,257mm},left=20mm,top=20mm,}
\pagenumbering{arabic}
\usepackage{amsmath,amssymb}
\usepackage{hyperref}
\usepackage{scrextend}
\usepackage{listings}
\setcounter{tocdepth}{1}
%\usepackage[backend=biber,style=nature,sorting=nyt]{biblatex}
%\addbibresource{su4.bib}


\begin{document}

\title{Classical Monte Carlo}
\author{Francisco Monteiro de Oliveira Brito}
\date{\today}
\maketitle

\begin{abstract}
Rarely ever do analytic solutions exist to most problems in statistical physics. This might be because of the mathematical difficulty or to the sheer number of possible configurations of the system which impedes a straightforward approximation. Typically we are interested in extracting properties of condensed matter systems. These are composed of a large number of parts 
\end{abstract}

\section{Introduction}\paragraph{}


\end{document}