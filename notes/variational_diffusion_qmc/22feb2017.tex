\documentclass[10pt]{article}
\usepackage[T1]{fontenc}
\usepackage{graphicx}
\usepackage{bm}
\usepackage{geometry}
\usepackage{bbm}
\geometry{a4paper,total={170mm,257mm},left=20mm,top=20mm,}
\pagenumbering{arabic}
\usepackage{amsmath,amssymb}
\usepackage{hyperref}
\usepackage{scrextend}
\usepackage{listings}
\setcounter{tocdepth}{1}
%\usepackage[backend=biber,style=nature,sorting=nyt]{biblatex}
%\addbibresource{su4.bib}


\begin{document}

\title{Getting started with Quantum Monte Carlo}
\author{Francisco Monteiro de Oliveira Brito}
\date{\today}
\maketitle

\begin{abstract}
The whole is often more than just the sum of its parts. Unraveling complexity is about understanding what's within. Collective phenomena emerge as a result of the interactions between the individual components of a system. The properties of these components do not directly percolate up to the system; instead, they shape the interactions that dictate the system's properties sometimes in rather  unexpected ways. In particular, quantum many-body systems tend to show counter intuitive behavior. We will focus on a system of strongly correlated electrons. In principle, its properties can all be deduced from the solution of the (extremely complicated) many-fermion Schr\"odinger equation. However, for the majority of systems the resulting integrals have no analytic solution. This is due to the complications introduced by the coupling of identical quantum particles. There is a myriad of methods to evaluate a integrals numerically. How do we pick the best one for this case? It depends on the problem at hand, of course. Quantum Monte Carlo (QMC) is a generic name for a variety of techniques based on using a stochastic process to find reliable and unbiased solutions to this kind of problem. For the specific case of many-fermion systems, the sign problem deems the task even more challenging. The antisymmetric nature of the many-fermion wave function leads to a sign oscillation that greatly impedes the accurate evaluation of averages. Variational and Diffusion QMC (DQMC) serve an introduction to using QMC methods, but further refinement is required to achieve meaningful results for systems of strongly interacting electrons. DQMC can be regarded as the canvas on which we will start painting a specific method that is optimally adapted to our problem.
\end{abstract}

\section{Variational Monte Carlo}

Variational techniques rely on an educated guess for the wave function of the system. One often introduces a set of variational parameters $\bm \alpha$ that are then tuned according to a variational principle. We use the optimized trial wave function to compute physical quantities of interest using Monte Carlo. Note that the method requires  prior knowledge about the system. We will see that it is natural to combine the Metropolis algorithm for quantum many-body systems with the variational principle.

A particularly relevant observable is the variational energy $E_V$ associated with a given quantum state. Let $\bm r$ be the $3N$ spatial coordinates of the $N$ electrons. Given the Hamiltonian of the system $\mathcal{H}$, and a trial wave function $\psi (\bm r)$ - hopefully a good guess of the wave function representing some state of the system - one can compute the variational energy.

\begin{equation}\label{eq:variational_energy}
E_V = \frac{\left\langle \psi | \mathcal{H} | \psi \right \rangle}{\left\langle \psi | \psi \right \rangle} = \frac{ \int d\bm r |\psi (\bm r)|^2 E_L (\bm r)}{\int d\bm r | \psi (\bm r)|^2 } = \int d\bm r\rho (\bm r) E_L (\bm r) ,
\end{equation}
where the local energy $E_L (\bm r)$ is defined as

\begin{equation}\label{eq:local_energy}
E_L = \frac{\mathcal{H} \psi (\bm r) }{\psi (\bm r)}
\end{equation}
and the probability distribution $\rho (\bm r)$ is defined as

\begin{equation}\label{eq:rho}
\rho (\bm r) = \frac{ | \psi (\bm r) |^2}{ \int d\bm r' | \psi (\bm r') |^2}
\end{equation}

Note that we managed to recast the variational energy as an average  of the \emph{local} energy, $< E_L > $, over the the distribution $\rho$. This is easily estimated by sampling $M$ points $\bm r_k$ from $\rho (\bm r)$:

\begin{equation}\label{eq:average}
E_V \approx \overline E_L = \frac{1}{M} \sum_{k= 1}^{M} E_L (\bm r_k) ,
\end{equation}
where $\overline {X}$ denotes a sample mean of the random variable $X$. A common sampling scheme is the Metropolis-Hastings algorithm.

How do we optimize a variational state? Suppose we are particularly interested in the ground state energy $E_0$. Then, the variational principle reads

\begin{equation}
E_{0_V}[\alpha_i] = \frac{< \psi_0 | \mathcal{H} | \psi_0 >}{<\psi_0 | \psi_0>} \ge E_0,
\end{equation}
where $\psi_0$ is the ground state wave function. By varying $\alpha$ we aim to obtain a variational energy that is as close as possible to the true ground state energy. Since $E_{0_V}(\alpha)$ is bounded from below, this is equivalent to minimizing it in the hope that $E_{0_V}(\alpha_{min}) \gtrsim E_0$ 

The use of an approximate wave function introduces a systematic error. The finite sampling size $M$, however, introduces a statistical uncertainty common to all Monte Carlo methods. 

\end{document}